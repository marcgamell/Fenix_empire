% This is "sig-alternate.tex" V2.0 May 2012
% This file should be compiled with V2.5 of "sig-alternate.cls" May 2012
%
% This example file demonstrates the use of the 'sig-alternate.cls'
% V2.5 LaTeX2e document class file. It is for those submitting
% articles to ACM Conference Proceedings WHO DO NOT WISH TO
% STRICTLY ADHERE TO THE SIGS (PUBS-BOARD-ENDORSED) STYLE.
% The 'sig-alternate.cls' file will produce a similar-looking,
% albeit, 'tighter' paper resulting in, invariably, fewer pages.
%
% ----------------------------------------------------------------------------------------------------------------
% This .tex file (and associated .cls V2.5) produces:
%       1) The Permission Statement
%       2) The Conference (location) Info information
%       3) The Copyright Line with ACM data
%       4) NO page numbers
%
% as against the acm_proc_article-sp.cls file which
% DOES NOT produce 1) thru' 3) above.
%
% Using 'sig-alternate.cls' you have control, however, from within
% the source .tex file, over both the CopyrightYear
% (defaulted to 200X) and the ACM Copyright Data
% (defaulted to X-XXXXX-XX-X/XX/XX).
% e.g.
% \CopyrightYear{2007} will cause 2007 to appear in the copyright line.
% \crdata{0-12345-67-8/90/12} will cause 0-12345-67-8/90/12 to appear in the copyright line.
%
% ---------------------------------------------------------------------------------------------------------------
% This .tex source is an example which *does* use
% the .bib file (from which the .bbl file % is produced).
% REMEMBER HOWEVER: After having produced the .bbl file,
% and prior to final submission, you *NEED* to 'insert'
% your .bbl file into your source .tex file so as to provide
% ONE 'self-contained' source file.
%
% ================= IF YOU HAVE QUESTIONS =======================
% Questions regarding the SIGS styles, SIGS policies and
% procedures, Conferences etc. should be sent to
% Adrienne Griscti (griscti@acm.org)
%
% Technical questions _only_ to
% Gerald Murray (murray@hq.acm.org)
% ===============================================================
%
% For tracking purposes - this is V2.0 - May 2012

\documentclass{sig-alternate}

\def\sharedaffiliation{%
\end{tabular}
\begin{tabular}{c}}

\def\_{\char95\discretionary{}{}{}}

\usepackage[absolute]{textpos}

%% \setlength{\TPHorizModule}{\paperwidth}\setlength{\TPVertModule}{\paperheight}
%% \TPMargin{5pt}
%% \newcommand{\copyrightstatement}{
%%   \begin{textblock}{0.84}(0.072,0.93)
%%     \footnotesize
%%     \vspace{0.1cm}
%%     \noindent Conference copyright line 1\\
%%     Conference copyright line 2 \copyright2015 IEEE
%%   \end{textblock}
%% }



%\usepackage[labelformat=empty]{caption}
\usepackage{caption}

\usepackage{url}
\usepackage{graphicx}
\usepackage{graphics}
\usepackage{subfigure}
\usepackage{multicol}
\usepackage{multirow}
\usepackage{pifont}
\usepackage[sort]{cite}
\usepackage[table]{xcolor}
\usepackage{afterpage}
%% \usepackage{lmodern}
\usepackage{hyperref}
\usepackage{booktabs}
\usepackage{listings}
\usepackage[normalem]{ulem}
\usepackage{array}
\usepackage{multirow}
\usepackage{tabularx}
\usepackage{fancyvrb}
\usepackage{flushend}
\usepackage{amsmath}
%\usepackage{graphicx}
%\usepackage{caption}
%\usepackage{subcaption}

%\usepackage{enumitem}

\usepackage{mathtools}
\DeclarePairedDelimiter\ceil{\lceil}{\rceil}
\DeclarePairedDelimiter\floor{\lfloor}{\rfloor}
\newcommand{\regtm}{\textsuperscript{\textregistered{}}}

%% \setlist{nolistsep}

\def\D{\mathrm{d}}

\newcommand{\marcnote}[1]{{\color{red}{Marc: #1}}}
\newcommand{\todo}[1]{{\color{red}{TODO: #1}}}
%% \newcommand{\todo}[1]{{\color{red}{TODO:}}}
%% \newcommand{\marcnote}[1]{{\color{red}{Marc note: }}}


\newcommand{\marchidecontent}[1]{{}}


\newcommand{\para}[1]{{\vspace{0.2em}\noindent{\bf{#1.}}\hspace{0.2em}}}
\newcommand{\parau}[1]{{\vspace{0.2em}\noindent{\underline{\bf{#1.}}}\hspace{0.2em}}}

\graphicspath{figs/}

\newcommand\userinput[1]{\textbf{#1}}
%\renewcommand{\thesubfigure}{(\thefigure.\arabic{subfigure})}
%\renewcommand{\thesubfigure}{(\arabic{subfigure})}

%% \makeatletter
%% \def\@copyrightspace{\relax}
%% \makeatother



%% \usepackage[absolute]{textpos}

%% \setlength{\TPHorizModule}{\paperwidth}\setlength{\TPVertModule}{\paperheight}
%% \TPMargin{5pt}


%% %define \copyrightstatement command for easier use
%% \newcommand{\copyrightstatement}{
%%     \begin{textblock}{0.84}(0.072,0.93)    % tweak here: {box width}(leftposition, rightposition)
%% %    \begin{textblock}{0.84}(0.08,0.93)    % tweak here: {box width}(leftposition, rightposition)
%% %         \noindent
%%          \footnotesize
%% %
%% %\IEEEpubid{
%% %\makebox[\columnwidth]{
%% \vspace{0.1cm}

%% \noindent SC15, November 15-20, 2015, Austin, TX\\
%% 978-1-4799-5500-8/14/\$31.00~\copyright2015 IEEE
%% %IEEE \hfill} \hspace{\columnsep}\makebox[\columnwidth]{ }}
%% %
%% %         \copyright  IEEE, 2012. Blablablablabla
%%     \end{textblock}
%% }



%\hyphenation{lo-cal-ly-cre-ated}



\begin{document}

\setcopyright{usgovmixed}
\conferenceinfo{ExaMPI'16,}{November 13, 2016, Salt Lake City, UT, USA}
\isbn{978-1-4503-3723-6/15/11}\acmPrice{\$15.00}
\doi{http://dx.doi.org/10.1145/2807591.2807672}


%\copyrightstatement
%\conferenceinfo{Supercomputing 2015}{'15 Austin, Texas USA}
%
% --- Author Metadata here ---
%\conferenceinfo{HPDC}{'15 Portland, Oregon USA}
%\CopyrightYear{2007} % Allows default copyright year (20XX) to be over-ridden - IF NEED BE.
%\crdata{0-12345-67-8/90/01}  % Allows default copyright data (0-89791-88-6/97/05) to be over-ridden - IF NEED BE.
% --- End of Author Metadata ---

\title{Evaluating the Fenix MPI Fault Tolerance Programming Framework}

%\subtitle{[Extended Abstract]
%\titlenote{A full version of this paper is available as
%\textit{Author's Guide to Preparing ACM SIG Proceedings Using
%\LaTeX$2_\epsilon$\ and BibTeX} at
%\texttt{www.acm.org/eaddress.htm}}}
%
% You need the command \numberofauthors to handle the 'placement
% and alignment' of the authors beneath the title.
%
% For aesthetic reasons, we recommend 'three authors at a time'
% i.e. three 'name/affiliation blocks' be placed beneath the title.
%
% NOTE: You are NOT restricted in how many 'rows' of
% "name/affiliations" may appear. We just ask that you restrict
% the number of 'columns' to three.
%
% Because of the available 'opening page real-estate'
% we ask you to refrain from putting more than six authors
% (two rows with three columns) beneath the article title.
% More than six makes the first-page appear very cluttered indeed.
%
% Use the \alignauthor commands to handle the names
% and affiliations for an 'aesthetic maximum' of six authors.
% Add names, affiliations, addresses for
% the seventh etc. author(s) as the argument for the
% \additionalauthors command.
% These 'additional authors' will be output/set for you
% without further effort on your part as the last section in
% the body of your article BEFORE References or any Appendices.

\numberofauthors{1}
\author{
% You can go ahead and credit any number of authors here,
% e.g. one 'row of three' or two rows (consisting of one row of three
% and a second row of one, two or three).
%
% The command \alignauthor (no curly braces needed) should
% precede each author name, affiliation/snail-mail address and
% e-mail address. Additionally, tag each line of
% affiliation/address with \affaddr, and tag the
% e-mail address with \email.
%
% 1st. author
\alignauthor Rob F. Van Der Wijngaart$^{\P}$,
Marc Gamell$^{\P}$,
Keita Teranishi$^{\ddag}$, 
Eric Valenzuela$^{*}$, 
Michael A. Heroux$^{\ddag}$, 
Manish Parashar$^{\dag}$\\
%
\affaddr{
\vspace{0.2cm}
$^{\P}$Intel Corporation, Santa Clara, CA, USA\\
{\it rob.f.van.der.wijngaart,marc.gamell.balamana@intel.com}\\
$^{\dag}$%NSF Cloud and Autonomic Computing Center \& 
Rutgers Discovery Informatics Institute,
Rutgers University, Piscataway, NJ, USA\\ 
{\it \{parashar\}@cac.rutgers.edu}\\
\vspace{0.2cm}
$^{\ddag}$Sandia National Laboratories, Livermore, CA, Albuquerque, NM, USA,\\
{\it \{knteran, maherou\}@sandia.gov}\\
$^{*}$California State University, Channel Islands, CA, USA,\\
{\it \{evalen\}@sandia.gov}
}
}
% There's nothing stopping you putting the seventh, eighth, etc.
% author on the opening page (as the 'third row') but we ask,
% for aesthetic reasons that you place these 'additional authors'
% in the \additional authors block, viz.
%% \additionalauthors{Additional authors: John Smith (The Th{\o}rv{\"a}ld Group,
%% email: {\texttt{jsmith@affiliation.org}}) and Julius P.~Kumquat
%% (The Kumquat Consortium, email: {\texttt{jpkumquat@consortium.net}}).}
%% \date{30 July 1999}
% Just remember to make sure that the TOTAL number of authors
% is the number that will appear on the first page PLUS the
% number that will appear in the \additionalauthors section.

\maketitle
%\begin{abstract}
%\end{abstract}

% Abusing ACM categories to select the topical area for SC'15
% Please change based on SC's guideline
\category{1}{System Software}{Support for fault-tolerance and resilience}
\category{2}{Performance} {Resilience}
\keywords{MPI, Fault Tolerance, performance, ULFM}
%% % A category with the (minimum) three required fields
%% \category{H.4}{Information Systems Applications}{Miscellaneous}
%% %A category including the fourth, optional field follows...
%% \category{D.2.8}{Software Engineering}{Metrics}[complexity measures, performance measures]

%% \terms{Theory}

%% \keywords{ACM proceedings, \LaTeX, text tagging}
\section{Introduction}
An important objective of extreme-scale high performance computing is fault tolerance:
the ability of applications to recover from failures efficiently.
Traditional checkpoint/restart-based techniques face scaling hurdles due to decreasing
mean time between failures, increasing cost of checkpointing, and 
increasing application relaunch time.
Fenix is a framework enabling MPI applications to recover nearly transparently from 
losses of data and/or compute resources manifested as observable errors.
It is based on the premise that the MPI standard itself provide facilities for trapping
and isolating such errors, allowing the application to retain control of remaining unaffected
resources, obviating  full program restart.

The current implementation of the Fenix specification \cite{fenixspec} leverages
the  User Level Fault Mitigation (ULFM) \cite{bland2013post} proposal for extension
of the MPI standard.
A prototype implementation of the framework \cite{Gamell:2014} proved successful for
an important production-level simulation code, but concerns have been raised about
overheads incurred by Fenix and ULFM in the absence of failres, as well as of the
applicability of these tools for tightly coupled HPC applications.
In this paper we address both concerns through the use of the Parallel Research Kernels
\cite{van2016comparing}, and firmly conclude that ULFM adds no measurable
overhead to applications, not even at very fine granularity.
Moreover, we find that overheads of Fenix are very modest, and can be further reduced.
Finally, our experience shows that Fenix+ULFM are perfectly suitable for tightly coupled
HPC applications.

This paper is structured as follows.
In Section \ref{sec:implem} we describe the salient features of the Fenix and ULFM
implementations that potentially impact performance of MPI applications.
In Section \ref{sec:PRK} we describe the Parallel Research Kernels and the modifications
made to them to incorporate Fenix fault tolerance, including potential pitfalls.
In Section \ref{sec:results} we present measurement data for the kernels,
followed by a summary and conclusions in Section{sec:conclusions}.

\section{ULFM and Fenix}\label{sec:implem}
ULFM adds a small set of functions to the MPI standard.
They enable users to return a program to a consistent state after a failure occurs.
The ULFM functions of use to Fenix are MPI\_Comm\_shrink, MPI\_Comm\_revoke, and MPI\_Comm\_agree.
These functions are invoked only when an error occurs, and hence do
not add the the MPI critical path.
However, ULFM requires that each non-local MPI call check an error code to
monitor the occurrence of communication or communicator failures.
This overhead itself is strictly local, which means it is constant, regardless of the
size of the communicator involved, or the network distance of any communicating rank(s).
It is the only overhead that could affect performance of applications using
ULFM-enabled MPI.

Fenix is a library \cite{fenixspec} that conveniently supports MPI process,
as well as data recovery. In this paper we focus on the former.
Fenix is initialized with a call to \texttt{Fenix\_Init}, in which the user specifies,
among others, how many ranks should be kept in reserve, to be used to substitute for
defunct ranks after a failure; these ranks stay coralled inside texttt{Fenix\_Init}
until further notice.
Users also specify an input communicator (usually \texttt{MPI\_COMM\_WORLD}), as well
as a resilient output communicator.
If the latter is left blank, Fenix will create an anonymous one, and will tacitly
replace the communicator in any MPI call that matches the input communicator with
the anonymous output communicator.
When a failure occurs, control is transferred back to \texttt{Fenix\_Init}, which patches
the resilient communicator with reserved ranks, utilizing among others functions
defined by ULFM.
It also automatically deletes any communicators that were derived from that base
communicator, and removes pending communication requests.
These processes only occur in case of failure, so they do not increase MPI's critical
path. However, other operations that Fenix carries out do add to it.

Fenix uses the MPI profiling interface (PMPI) to
perform the following tasks whenever a non-trivial MPI call is made:
\begin{enumerate}
\item check whether the library has been initialized.\label{item:init}
\item check whether the communicator named in the operation should be replaced with
  the anonymous output communicator.\label{item:anon}
\item check whether the MPI call has generated an error code corresponding to a
  communication or communicator failure.\label{item:procerr}
\item record a newly derived communicator in an internally maintained data structure.\label{item:derive}
\item record any created communication request in an internally maintained data structure.\label{item:record}
\item remove such a requestt from that data structure upon completion of the request.\label{item:clear}
\end{enumerate}
Of these, task \ref{item:derive} is normally not important for performance, since
in most HPC codes communicator creation, which is relatively expensive, is infrequent.
Tasks \ref{item:init}--\ref{item:procerr} are simple scalar tests of one statement each.
Tasks \ref{item:record} and \ref{item:clear} require accessing non-trivial data structures,
which could be more time-consuming.



\section{The Parallel Research Kernels and Fenix}\label{sec:PRK}
The Parallel Research Kernels \cite{van2016comparing,PRKrepo}
are a suite of simple kernels designed to stress various aspects of parallel computing systems.
We select the following subset of the kernels for the evaluation of Fenix.
Stencil, an explicit, data-parallel kernel that applies a simple 9-point star stencil
operation to a scalar input array and accumulates the results in an output array; it is nominally medium grain.
Transpose, a matrix transposition kernel in which a square matrix is divided into strips among the
participating ranks, which subsequently must be transposed, and accumulated at the destination ranks using
the same decomposition, requiring an all-to-all communication pattern;
its granularity ranges from quite fine to coarse.
Synch\_p2p, a semi-implicit stencil kernel that requires very frequent, short messages, except for
the very largest grids.
AMR, an adaptive mesh refinement kernel \cite{AMRPRK} that experiences periodic introduction of
a refinement grid that requires some localized work; it is the same as the Stencil kernel,
but with the added complexity of the refinements.

These four kernels represent common patterns in scientific HPC applications. Most partial differential
equation solvers use patterns like Stencil and/or Synch\_p2p. Transpose proxies the data movement
in Fast Fourier Transforms and other convolutions in particular, and global data redistribution in general.
AMR adds more realism to Stencil, and can be viewed as a proxy for S3D
\cite{Gamell:2014}.

We turned these kernels into testbeds for Fenix by adding the following logic.
All kernels do a number of nominally identical iterations, although the output arrays keep
changing their values in subsequent iterations, due to the accumulation in the case of Stencil,
Transpose, and AMR, and through a coupling of extremal grid values for Synch\_p2p
(see \cite{van2016comparing}).
The user specifies for each kernel the average number of iterations between rank failures,
the number of ranks that fail during each failure event, and the number of ranks to be
reserved by Fenix.
The kernel uses a pseudo-random number generator (repeatable) to compute actual iteration
numbers in which failures will be triggered.
Each active rank watches for these specific iterations, and, when triggered, sends a
kill signal to itself if it is within the set of ranks that should die.
Control is then returned to \texttt{Fenix\_Init} by the library and the damaged communicator(s)
are repaired.
At this stage we are not concerned with the data recovery utilities of Fenix (these will be
added and evaluated later), so we use the analytical solutions that exist for all the PRK to (re-)initialize
all ranksd, including replaced ranks, after a failure.



\section{Experimental results}\label{sec:results}
We measured performance of the PRK on a shared memory workstation
equipped with two 18-core
Intel$\footnotesize{^{\textregistered}}$
Xeon$\footnotesize{^{\textregistered}}$ E5-2699 processors at 2.30GHz,
with total memory of 64 GB.
In all cases we used the Intel\regtm{} C compiler icc
version 18.0.0, Intel\regtm{} MPI Library for Linux version 5.0.
We note that for the purpose of this investigation--measuring runtime
overheads when no failures occur--a shared memory system
presents the toughest challenge for the fault-tolerance enhanced runtimes.
Overheads introduced by ULFM and Fenix are strictly local, and hence
are fixed, regardless of system size.
They are added to the communication costs incurred by MPI calls, which are
very small on a shared memory system.
Consequently, the effect of the overheads is magnified.

We compare four different runtime configurations.
At present the only MPI version that offers a robust implementation of ULFM
is OpenMPI \cite{openmpi}.
It can be built without fault tolerance (i.e. no overhead, label ``OpenMPI''), and with
fault tolerance (ULFM is enabled, label ``ULFM'').
In addition, it can be linked with the Fenix library (label ``Fenix'').
We note that the version of OpenMPI that accommodates ULFM is not the latest production
version.
Finally, we also do runs with the Intel\regtm{} MPI library (for Linux, version 5.0),
which is presumed to perform best of all MPI runtimes on an Intel-based system.
All kernels operate on 2D (distributed) arrays that correspond to 2D grids or matrices.
We pick a the following set of array sizes: 900, 1800, 3600, 7200, 14400, 28800,
57600.
For AMR we also need to pick a collection of sizes for the refinements.
We set those to: 450, 900, 1800, 3600, 7200, and note that the largest background
grid size of $57600^2$ points, combined with a refinement grid of $28800^2$
points, does not fit in the workstation's memory, so we do not present results
for that size.





%\section{Related work}\label{sec:related}
Reinit \cite{laguna2016evaluating} functionality is similar to 
Fenix', with these exceptions:
it assumes non-shrinking recovery, whereas Fenix supports shrinking and 
non-shrinking recovery within the same framework;
it offers no facilities for data recovery;
it is built directly into the MPI runtime, whereas Fenix is a library built
on top of MPI--the current implementation uses ULFM, which our specification
does not expose;
it changes program structure, replacing the original \texttt{main} 
with calls to cleanup handling, whereas Fenix-enabled codes retain 
their original structure--they can skip all fault tolerance
constructs via selective compilation or runtime tests;
it supports direct use of \texttt{MPI\_COMM\_WORLD}, but to accomplish
the same with Fenix requires the PMPI profiling interface. This prohibits use 
of other PMPI tools together with Fenix. 
Efforts within the MPI Forum target PMPI alternatives 
supporting multiple tools simultaneously.

Fault-Aware MPI \cite{Hassani15FAMPI} offers tunable  resilience
by introduction of transactions around user-specified code blocks.
It supports only asynchronous MPI communications.

Adaptive MPI, leveraging Charm++'s \cite{charm++} runtime, supports shrinking 
and non-shrinking recovery. 
It differs from Fenix in that it does not use a production
MPI runtime; it always maintains the same number of MPI ranks (implemented
as user-level threads), but redistributes those among remaining resources in
a shrinking recovery; it assumes and integrates data recovery through rollback, 
whereas Fenix decouples process and data recovery.

Without further reference we point to LFLR (Local Failure, Local Recovery) and
RTS (Run-Through Stabilization) as precursors to Fenix and ULFM, respectively.

\section{Conclusions}
Our experiments show that even for very fine-grain HPC codes, overheads
incurred by ULFM and Fenix over the baseline OpenMPI implementation
are negligible in the absence of failures.
Moreover, they demonstrate that Fenix is applicable to tightly-coupled,
bulkp-synchronous applications that are common in high-performance
scientific computing.
Slightly modified versions of the Parallel Research Kernels were
instrumental in drawing these conclusions.

%\newpage

%
% The following two commands are all you need in the
% initial runs of your .tex file to
% produce the bibliography for the citations in your paper.
\bibliographystyle{abbrv}
\bibliography{resilience_etal}
% You must have a proper ".bib" file
%  and remember to run:
% latex bibtex latex latex
% to resolve all references
%
% ACM needs 'a single self-contained file'!
%
%APPENDICES are optional
%\balancecolumns
%% \appendix
%% %Appendix A
%% \section{Headings in Appendices}
%% The rules about hierarchical headings discussed above for
%% the body of the article are different in the appendices.
%% In the \textbf{appendix} environment, the command
%% \textbf{section} is used to
%% indicate the start of each Appendix, with alphabetic order
%% designation (i.e. the first is A, the second B, etc.) and
%% a title (if you include one).  So, if you need
%% hierarchical structure
%% \textit{within} an Appendix, start with \textbf{subsection} as the
%% highest level. Here is an outline of the body of this
%% document in Appendix-appropriate form:
%% \subsection{Introduction}
%% \subsection{The Body of the Paper}
%% \subsubsection{Type Changes and  Special Characters}
%% \subsubsection{Math Equations}
%% \paragraph{Inline (In-text) Equations}
%% \paragraph{Display Equations}
%% \subsubsection{Citations}
%% \subsubsection{Tables}
%% \subsubsection{Figures}
%% \subsubsection{Theorem-like Constructs}
%% \subsubsection*{A Caveat for the \TeX\ Expert}
%% \subsection{Conclusions}
%% \subsection{Acknowledgments}
%% \subsection{Additional Authors}
%% This section is inserted by \LaTeX; you do not insert it.
%% You just add the names and information in the
%% \texttt{{\char'134}additionalauthors} command at the start
%% of the document.
%% \subsection{References}
%% Generated by bibtex from your ~.bib file.  Run latex,
%% then bibtex, then latex twice (to resolve references)
%% to create the ~.bbl file.  Insert that ~.bbl file into
%% the .tex source file and comment out
%% the command \texttt{{\char'134}thebibliography}.
%% % This next section command marks the start of
%% % Appendix B, and does not continue the present hierarchy
%% \section{More Help for the Hardy}
%% The sig-alternate.cls file itself is chock-full of succinct
%% and helpful comments.  If you consider yourself a moderately
%% experienced to expert user of \LaTeX, you may find reading
%% it useful but please remember not to change it.
%% %\balancecolumns % GM June 2007
%% % That's all folks!
\end{document}

% This is "sig-alternate.tex" V2.0 May 2012
% This file should be compiled with V2.5 of "sig-alternate.cls" May 2012
%
% This example file demonstrates the use of the 'sig-alternate.cls'
% V2.5 LaTeX2e document class file. It is for those submitting
% articles to ACM Conference Proceedings WHO DO NOT WISH TO
% STRICTLY ADHERE TO THE SIGS (PUBS-BOARD-ENDORSED) STYLE.
% The 'sig-alternate.cls' file will produce a similar-looking,
% albeit, 'tighter' paper resulting in, invariably, fewer pages.
%
% ----------------------------------------------------------------------------------------------------------------
% This .tex file (and associated .cls V2.5) produces:
%       1) The Permission Statement
%       2) The Conference (location) Info information
%       3) The Copyright Line with ACM data
%       4) NO page numbers
%
% as against the acm_proc_article-sp.cls file which
% DOES NOT produce 1) thru' 3) above.
%
% Using 'sig-alternate.cls' you have control, however, from within
% the source .tex file, over both the CopyrightYear
% (defaulted to 200X) and the ACM Copyright Data
% (defaulted to X-XXXXX-XX-X/XX/XX).
% e.g.
% \CopyrightYear{2007} will cause 2007 to appear in the copyright line.
% \crdata{0-12345-67-8/90/12} will cause 0-12345-67-8/90/12 to appear in the copyright line.
%
% ---------------------------------------------------------------------------------------------------------------
% This .tex source is an example which *does* use
% the .bib file (from which the .bbl file % is produced).
% REMEMBER HOWEVER: After having produced the .bbl file,
% and prior to final submission, you *NEED* to 'insert'
% your .bbl file into your source .tex file so as to provide
% ONE 'self-contained' source file.
%
% ================= IF YOU HAVE QUESTIONS =======================
% Questions regarding the SIGS styles, SIGS policies and
% procedures, Conferences etc. should be sent to
% Adrienne Griscti (griscti@acm.org)
%
% Technical questions _only_ to
% Gerald Murray (murray@hq.acm.org)
% ===============================================================
%
% For tracking purposes - this is V2.0 - May 2012

\documentclass{sig-alternate}

\def\sharedaffiliation{%
\end{tabular}
\begin{tabular}{c}}

\def\_{\char95\discretionary{}{}{}}

\usepackage[absolute]{textpos}

%% \setlength{\TPHorizModule}{\paperwidth}\setlength{\TPVertModule}{\paperheight}
%% \TPMargin{5pt}
%% \newcommand{\copyrightstatement}{
%%   \begin{textblock}{0.84}(0.072,0.93)
%%     \footnotesize
%%     \vspace{0.1cm}
%%     \noindent Conference copyright line 1\\
%%     Conference copyright line 2 \copyright2015 IEEE
%%   \end{textblock}
%% }



%\usepackage[labelformat=empty]{caption}
\usepackage{caption}

\usepackage{url}
\usepackage{graphicx}
\usepackage{graphics}
\usepackage{subfigure}
\usepackage{multicol}
\usepackage{multirow}
\usepackage{pifont}
\usepackage[sort]{cite}
\usepackage[table]{xcolor}
\usepackage{afterpage}
%% \usepackage{lmodern}
\usepackage{hyperref}
\usepackage{booktabs}
\usepackage{listings}
\usepackage[normalem]{ulem}
\usepackage{array}
\usepackage{multirow}
\usepackage{tabularx}
\usepackage{fancyvrb}
\usepackage{flushend}
\usepackage{amsmath}
%\usepackage{graphicx}
%\usepackage{caption}
%\usepackage{subcaption}

%\usepackage{enumitem}

\usepackage{mathtools}
\DeclarePairedDelimiter\ceil{\lceil}{\rceil}
\DeclarePairedDelimiter\floor{\lfloor}{\rfloor}


%% \setlist{nolistsep}

\def\D{\mathrm{d}}

\newcommand{\marcnote}[1]{{\color{red}{Marc: #1}}}
\newcommand{\todo}[1]{{\color{red}{TODO: #1}}}
%% \newcommand{\todo}[1]{{\color{red}{TODO:}}}
%% \newcommand{\marcnote}[1]{{\color{red}{Marc note: }}}


\newcommand{\marchidecontent}[1]{{}}


\newcommand{\para}[1]{{\vspace{0.2em}\noindent{\bf{#1.}}\hspace{0.2em}}}
\newcommand{\parau}[1]{{\vspace{0.2em}\noindent{\underline{\bf{#1.}}}\hspace{0.2em}}}

\graphicspath{figs/}

\newcommand\userinput[1]{\textbf{#1}}
%\renewcommand{\thesubfigure}{(\thefigure.\arabic{subfigure})}
%\renewcommand{\thesubfigure}{(\arabic{subfigure})}

%% \makeatletter
%% \def\@copyrightspace{\relax}
%% \makeatother



%% \usepackage[absolute]{textpos}

%% \setlength{\TPHorizModule}{\paperwidth}\setlength{\TPVertModule}{\paperheight}
%% \TPMargin{5pt}


%% %define \copyrightstatement command for easier use
%% \newcommand{\copyrightstatement}{
%%     \begin{textblock}{0.84}(0.072,0.93)    % tweak here: {box width}(leftposition, rightposition)
%% %    \begin{textblock}{0.84}(0.08,0.93)    % tweak here: {box width}(leftposition, rightposition)
%% %         \noindent
%%          \footnotesize
%% %
%% %\IEEEpubid{
%% %\makebox[\columnwidth]{
%% \vspace{0.1cm}

%% \noindent SC15, November 15-20, 2015, Austin, TX\\
%% 978-1-4799-5500-8/14/\$31.00~\copyright2015 IEEE
%% %IEEE \hfill} \hspace{\columnsep}\makebox[\columnwidth]{ }}
%% %
%% %         \copyright  IEEE, 2012. Blablablablabla
%%     \end{textblock}
%% }



%\hyphenation{lo-cal-ly-cre-ated}



\begin{document}

\setcopyright{usgovmixed}
\conferenceinfo{ExaMPI'16,}{November 13, 2016, Salt Lake City, UT, USA}
\isbn{978-1-4503-3723-6/15/11}\acmPrice{\$15.00}
\doi{http://dx.doi.org/10.1145/2807591.2807672}


%\copyrightstatement
%\conferenceinfo{Supercomputing 2015}{'15 Austin, Texas USA}
%
% --- Author Metadata here ---
%\conferenceinfo{HPDC}{'15 Portland, Oregon USA}
%\CopyrightYear{2007} % Allows default copyright year (20XX) to be over-ridden - IF NEED BE.
%\crdata{0-12345-67-8/90/01}  % Allows default copyright data (0-89791-88-6/97/05) to be over-ridden - IF NEED BE.
% --- End of Author Metadata ---

\title{Evaluating the Fenix MPI Fault Tolerance Programming Framework}

%\subtitle{[Extended Abstract]
%\titlenote{A full version of this paper is available as
%\textit{Author's Guide to Preparing ACM SIG Proceedings Using
%\LaTeX$2_\epsilon$\ and BibTeX} at
%\texttt{www.acm.org/eaddress.htm}}}
%
% You need the command \numberofauthors to handle the 'placement
% and alignment' of the authors beneath the title.
%
% For aesthetic reasons, we recommend 'three authors at a time'
% i.e. three 'name/affiliation blocks' be placed beneath the title.
%
% NOTE: You are NOT restricted in how many 'rows' of
% "name/affiliations" may appear. We just ask that you restrict
% the number of 'columns' to three.
%
% Because of the available 'opening page real-estate'
% we ask you to refrain from putting more than six authors
% (two rows with three columns) beneath the article title.
% More than six makes the first-page appear very cluttered indeed.
%
% Use the \alignauthor commands to handle the names
% and affiliations for an 'aesthetic maximum' of six authors.
% Add names, affiliations, addresses for
% the seventh etc. author(s) as the argument for the
% \additionalauthors command.
% These 'additional authors' will be output/set for you
% without further effort on your part as the last section in
% the body of your article BEFORE References or any Appendices.

\numberofauthors{1}
\author{
% You can go ahead and credit any number of authors here,
% e.g. one 'row of three' or two rows (consisting of one row of three
% and a second row of one, two or three).
%
% The command \alignauthor (no curly braces needed) should
% precede each author name, affiliation/snail-mail address and
% e-mail address. Additionally, tag each line of
% affiliation/address with \affaddr, and tag the
% e-mail address with \email.
%
% 1st. author
\alignauthor Rob F. Van Der Wijngaart$^{\P}$,
Marc Gamell$^{\P}$,
Keita Teranishi$^{\ddag}$, 
Eric Valenzuela$^{*}$, 
Michael A. Heroux$^{\ddag}$, 
Manish Parashar$^{\dag}$\\
%
\affaddr{
\vspace{0.2cm}
$^{\P}$Intel Corporation, Santa Clara, CA, USA\\
{\it rob.f.van.der.wijngaart,marc.gamell.balamana@intel.com}\\
$^{\dag}$%NSF Cloud and Autonomic Computing Center \& 
Rutgers Discovery Informatics Institute,
Rutgers University, Piscataway, NJ, USA\\ 
{\it \{parashar\}@cac.rutgers.edu}\\
\vspace{0.2cm}
$^{\ddag}$Sandia National Laboratories, Livermore, CA, Albuquerque, NM, USA,\\
{\it \{knteran, maherou\}@sandia.gov}\\
$^{*}$California State University, Channel Islands, CA, USA,\\
{\it \{evalen\}@sandia.gov}
}
}
% There's nothing stopping you putting the seventh, eighth, etc.
% author on the opening page (as the 'third row') but we ask,
% for aesthetic reasons that you place these 'additional authors'
% in the \additional authors block, viz.
%% \additionalauthors{Additional authors: John Smith (The Th{\o}rv{\"a}ld Group,
%% email: {\texttt{jsmith@affiliation.org}}) and Julius P.~Kumquat
%% (The Kumquat Consortium, email: {\texttt{jpkumquat@consortium.net}}).}
%% \date{30 July 1999}
% Just remember to make sure that the TOTAL number of authors
% is the number that will appear on the first page PLUS the
% number that will appear in the \additionalauthors section.

\maketitle
%\begin{abstract}
%\end{abstract}

% Abusing ACM categories to select the topical area for SC'15
% Please change based on SC's guideline
\category{1}{System Software}{Support for fault-tolerance and resilience}
\category{2}{Performance} {Resilience}
\keywords{MPI, Fault Tolerance, performance, ULFM}
%% % A category with the (minimum) three required fields
%% \category{H.4}{Information Systems Applications}{Miscellaneous}
%% %A category including the fourth, optional field follows...
%% \category{D.2.8}{Software Engineering}{Metrics}[complexity measures, performance measures]

%% \terms{Theory}

%% \keywords{ACM proceedings, \LaTeX, text tagging}
\section{Introduction}
An important objective of extreme-scale high performance computing is fault tolerance:
the ability of applications to recover from failures efficiently.
Traditional checkpoint/restart-based techniques face scaling hurdles due to decreasing
mean time between failures, increasing cost of checkpointing, and 
increasing application relaunch time.
Fenix is a framework enabling MPI applications to recover nearly transparently from 
losses of data and/or compute resources manifested as observable errors.
It is based on the premise that the MPI standard itself provide facilities for trapping
and isolating such errors, allowing the application to retain control of remaining unaffected
resources, obviating full program restart.

The current implementation of the Fenix specification \cite{fenixspec} leverages
the  User Level Fault Mitigation (ULFM) \cite{bland2013post} proposal for extension
of the MPI standard.
A prototype implementation of the framework \cite{Gamell:2014} proved successful for
an important production-level simulation code, but concerns have been raised about
overheads incurred by Fenix and ULFM in the absence of failures, as well as of the
applicability of these tools for bulk-synchronous HPC applications.
In this paper we address both concerns through the use of the Parallel Research Kernels
\cite{van2016comparing}, and firmly conclude that ULFM adds no measurable
overhead to applications, not even at very fine granularity.
Moreover, we find that overheads of Fenix are very modest, and can be further reduced.
Finally, our experience shows that Fenix+ULFM are perfectly suitable for tightly coupled
HPC applications.

This paper is structured as follows.
In Section \ref{sec:implem} we briefly describe the salient features of Fenix and ULFM,
and describe the implementation factors that potentially impact performance of MPI applications.
In Section \ref{sec:PRK} we describe the Parallel Research Kernels and the modifications
made to them to incorporate Fenix fault tolerance, including potential pitfalls.
In Section \ref{sec:results} we present measurement data for the kernels,
followed by a summary and conclusions in Section{sec:conclusions}.

%\section{Fenix specification}\label{sec:spec}
Fenix has two distinct interfaces: process recovery, and data recovery.
The first allows an MPI application to recover from a permanent loss of MPI 
processes (ranks) that cause MPI calls to fail.
This is the most important and novel part of Fenix.
Fenix' data recovery API could be replaced by, or used with, other mechanisms to 
restore application data.
\subsection{Process recovery}
Fenix' basic assumption is that nominally fatal errors in MPI programs
are detected by the runtime and reported via error codes.
While default error response is application shutdown
(\texttt{MPI\_ERRORS\_ARE\_FATAL}), Fenix overrides this, allowing remaining
resources (MPI processes) to be informed of the failure, and to call MPI functions 
to remove the lost resources from their respective communication contexts \cite{Gamell:2014}.

To ease adoption of MPI fault tolerance, Fenix automatically captures 
errors resulting from MPI library calls that experience a failure. 
%
Implementations of the Fenix specification can achieve this behavior
leveraging error handlers or the MPI profiling interface.
%
Hence, Fenix users need \emph{not} replace MPI calls with calls to Fenix  
(for example,\texttt{Fenix\_Send} instead of \texttt{MPI\_Send}).
Subsequently, Fenix repairs communicators transparently and returns  execution 
control to the application. 
Its detailed behavior is determined by function \texttt{Fenix\_Init}, which
also initializes the library.
\begin{verbatim}
void Fenix_Init(
   MPI_Comm comm,
   MPI_Comm *newcomm, 
   int *role, 
   int *argc, char ***argv, 
   int spare_ranks, 
   int spawn,
   MPI_Info info,
   int *error);
\end{verbatim}
Communicator \texttt{comm} (includes any slack)
is used to create \textit{resilient} communicator \texttt{newcomm}. 
Communicators derived from \texttt{newcomm} are automatically resilient.
Errors returned by MPI calls involving resilient communicators are intercepted by Fenix,
triggering repair of all resilient communicators, such that the application can resume execution.
Control is returned to the application at the 
logical exit of \texttt{Fenix\_Init}.

Parameter \texttt{role} contains the calling rank's most recent history.
Possible values are \texttt{FENIX\_ROLE\_\textit{X}\_RANK},
with \texttt{\textit{X}} being \texttt{INITIAL}, \texttt{SURVIVOR}, or \texttt{RECOVERED},
corresponding to ``no errors yet'', ``not affected by latest failure'', 
and ``rank recovered, but has no useful application data'', respectively.

Parameter \texttt{spare\_ranks} specifies how many ranks in \texttt{comm} are 
sequestered in \texttt{Fenix\_Init}  to replace failed ranks, and
\texttt{spawn} determines whether Fenix  may create new ranks (\texttt{MPI\_Comm\_spawn}) 
to restore resilient communicators to their original size. 
\texttt{spare\_ranks} and \texttt{spawn} together define Fenix' overall communicator
repair policy. 
If both are zero, resilient communicators are shrunk to exclude failed ranks after 
an error.
If only \texttt{spawn} is zero, Fenix draws on the spare 
ranks pool to restore resilient communicators to their original size. 
Once the pool is depleted, subsequent errors lead to communicator shrinkage.
Parameter \texttt{info} conveys details about expected Fenix behavior
that differs from the default.
%as well as to allow \texttt{MPI\_Comm\_spawn} configuration.
%{\color{red}Should we include an explanation of newcomm=NULL to make comm resilient through PMPI?}

\begin{verbatim}
int Fenix_Callback_register(
   void (*recover)(MPI_Comm, int, void*),
   void *callback_data);
\end{verbatim}
This function registers a callback function, to be invoked by \texttt{SURVIVOR}
ranks after a failure has been recovered by Fenix, and right before resuming application
execution.
Multiple callback functions may be registered.

\subsection{Data recovery}
Once Fenix process recovery has returned an application to a consistent state, the user 
needs to consider lost data.
Sometimes no action is required, for example because the application
is embarrassingly parallel Monte Carlo.
However, in most HPC applications,  Fenix' prime target, ranks synchronize often, 
and intermediate state of ranks lost due to error must be restored.
We outline several plausible approaches, all supported by Fenix'
process and data recovery facilities.
However, the user  need not use Fenix for data recovery, and may, for example,
use Global View Resilience \cite{chien2015versioned}, Scalable Checkpoint Restart
\cite{moody2010detailed}, a combination of these and Fenix, etc.
Fenix aims mostly at providing fast, in-memory redundant storage for data recovery,
whereas GVR and SCR target file storage, where space is a much less scarce.
\begin{enumerate}
\item \textit{Non-shrinking recovery with full data retrieval.} \label{1}
This is the most common case in bulk-synchronous HPC codes. 
The programmer defines data/work decomposition that corresponds to a certain
number of ranks.
After an error, \texttt{SURVIVOR}s roll back their state to a prior time.
Missing ranks are replaced with \texttt{RECOVERED} ranks who instantiate their
state using non-local data retrieved by a simple Fenix function invocation.
Time-accurate computations favor this approach.
\item \textit{Non-shrinking recovery with local data retrieval only.}\label{2}
\texttt{SURVIVOR}s roll back their state, but 
\texttt{RECOVERED} ranks approximate their requisite data,
for example by interpolation of logically ``nearby'' data.
This approach, also good for bulk-synchronous codes, may apply to relaxation methods.
\item \textit{Shrinking recovery with full data retrieval.}\label{3}
If the user demands online recovery and
resources are insufficient to replace defunct ranks, 
Fenix shrinks damaged communicators.
Now there is no simple, unique way to 
assign recovered data to the reduced number of remaining ranks.
More general, flexible Fenix data recovery functions are provided for alternate
ways of retrieving and re-assigning such data.
\item \textit{Shrinking recovery with local data retrieval only.}\label{4}
This is a combination of methods \ref{2} and \ref{3}.
\end{enumerate}

To organize redundant storage for data recovery after a fault,
Fenix offers \textit{data groups}, containers for sets of data objects 
(\textit{members}) that are manipulated as a unit.
Data groups also refer to the collection of ranks that
cooperate in handling recovery data.
This collection need not include all active ranks.
Fenix adopts the convenient MPI vehicle of \textit{communicators} to
indicate the subset of ranks involved.

A data group is instantiated with the following function.
\begin{verbatim}
int Fenix_Data_group_create(
   int group_id,
   MPI_Comm comm, 
   int start_time_stamp, 
   int depth);
\end{verbatim}

It is called by all ranks in \texttt{comm}--with identical 
parameter values--defining the group for all of them. 
The label \texttt{group\_id} can be used to restore the group after a failure.
Parameter \texttt{start\_time\_stamp} initializes a counter to identify versions
of the recovery data associated with the data group, and \texttt{depth} is
the number of successive versions of such data sets maintained by Fenix
in addition to the latest.
%Subsequent operations on the group are all collective over \texttt{comm}, but not
%necessarily synchronizing.

Once a data group has been created, the user can define its members, which describe
the actual application data:
\begin{verbatim}
int Fenix_Data_member_create( 
   int group_id, 
   int member_id,
   void *source_buffer, 
   int count, 
   MPI_Datatype datatype);
\end{verbatim}
The \texttt{member\_id} distinguishes between different group 
members, \texttt{source\_buffer} is the address of the contiguous application data in memory,
and \texttt{count} (may be different for different ranks) and \texttt{datatype} together 
fix the data's extent.

Once all group members have been defined, the user can invoke Fenix functions to store
application data:
\begin{verbatim}
int Fenix_Data_member_store( 
   int group_id, 
   int member_id, 
   Fenix_Data_subset subset_specifier);
\end{verbatim}
Set \texttt{member\_id} to  \texttt{FENIX\_DATA\_MEMBER\_ALL}
to store all group members.
Parameter \texttt{subset\_specifier} controls selective storage.
%value \texttt{FENIX\_DATA\_SUBSET\_FULL} causes the entire member to be stored.
Currently Fenix stores two copies of each member;
one in the memory of the calling rank, the other in a peer rank's in \texttt{comm}.
Upon failure \texttt{SURVIVOR}s restore their
application data using the local copy, and \texttt{RECOVERED} ranks 
fetch it from their peer. This technique, inspired by
Charm++ \cite{charm++}, provides resilience through redundancy.
%The level of resilience can be influenced by the 
%location of the peer rank relative to the calling rank.
%The latter may be set via a \textit{separation distance}, where a
%peer's rank equals the calling rank plus the separation distance, modulo
%communicator size. 
%The default separation distance is half of the communicator size; this value
%can be changed using a redundancy policy accessor function.
%With contiguous ranks placement, a separation distance equal to the number 
%of cores on a multi-core node nominally puts a rank's Fenix peer on another node.
%provided the communicator spans multiple nodes.
%This would allow a whole node to fail and still allow 
%recovery via in-memory redundant data storage.

To mark a version of the group's data as recoverable,
it needs to be committed; Fenix labels the stored data with a time
stamp (a \textit{snapshot}).
Subsequent store operations of the same  group do not contribute
to the same snapshot, and receive an incremented time stamp upon the next commit.
\begin{verbatim}
int Fenix_Data_commit(
   int group_id,
   int *time_stamp);
\end{verbatim}

When an error occurs and process recovery is non-shrinking, application data
for a particular group member can be restored simply using the following collective function.
\begin{verbatim}
int Fenix_Data_member_restore(
   int group_id, 
   int member_id,
   void *target_buffer, 
   int max_count, 
   int time_stamp);
\end{verbatim}
Upon return each rank in the repaired communicator can access the
extracted snapshot at address \texttt{target\_buffer}, if
it is within \texttt{max\_count} units of the member's MPI data type.

When an error occurs and process recovery is shrinking, there will be fewer
ranks after the failure than before. This case is supported by a more
general data recovery function.
\begin{verbatim}
int Fenix_Data_member_restore_from_rank(
   int group_id, 
   int member_id, 
   void *target_buffer, 
   int max_count, 
   int time_stamp,
   int source_rank);
\end{verbatim}
This collective function names the source rank (pre-error) explicitly.
Let the size of the affected communicator before and after the failure be $C_b$ and $C_a$,
respectively. The calling rank is $R$.
Calling \texttt{restore} twice, with \texttt{source\_rank} equal to $R$ and
$C_a+R$, respectively, returns all application data, provided ranks
$C_b-C_a$ through $C_a-1$ set \texttt{max\_count} to zero in the second
round to avoid searching for non-existent data.
Next, data needs to be redistributed among the ranks to avoid load imbalance.
This is beyond the scope of Fenix.

%In addition to the above, Fenix provides query functions, 
%non-blocking  storage functions to improve performance,
%various synchronization and implicit and explicit garbage collection functions, and functions to 
%manipulate subsets.

In addition to the above, Fenix provides query, synchronization, and
implicit and explicit garbage collection functions, as well as non-blocking  
storage functions to improve performance, and functions to manipulate subsets.

%\section{Related work}\label{sec:related}
Reinit \cite{laguna2016evaluating} functionality is similar to 
Fenix', with these exceptions:
it assumes non-shrinking recovery, whereas Fenix supports shrinking and 
non-shrinking recovery within the same framework;
it offers no facilities for data recovery;
it is built directly into the MPI runtime, whereas Fenix is a library built
on top of MPI--the current implementation uses ULFM, which our specification
does not expose;
it changes program structure, replacing the original \texttt{main} 
with calls to cleanup handling, whereas Fenix-enabled codes retain 
their original structure--they can skip all fault tolerance
constructs via selective compilation or runtime tests;
it supports direct use of \texttt{MPI\_COMM\_WORLD}, but to accomplish
the same with Fenix requires the PMPI profiling interface. This prohibits use 
of other PMPI tools together with Fenix. 
Efforts within the MPI Forum target PMPI alternatives 
supporting multiple tools simultaneously.

Fault-Aware MPI \cite{Hassani15FAMPI} offers tunable  resilience
by introduction of transactions around user-specified code blocks.
It supports only asynchronous MPI communications.

Adaptive MPI, leveraging Charm++'s \cite{charm++} runtime, supports shrinking 
and non-shrinking recovery. 
It differs from Fenix in that it does not use a production
MPI runtime; it always maintains the same number of MPI ranks (implemented
as user-level threads), but redistributes those among remaining resources in
a shrinking recovery; it assumes and integrates data recovery through rollback, 
whereas Fenix decouples process and data recovery.

Without further reference we point to LFLR (Local Failure, Local Recovery) and
RTS (Run-Through Stabilization) as precursors to Fenix and ULFM, respectively.

%\section{Future work}\label{sec:future}
%\newpage

%
% The following two commands are all you need in the
% initial runs of your .tex file to
% produce the bibliography for the citations in your paper.
\bibliographystyle{abbrv}
\bibliography{resilience_etal}
% You must have a proper ".bib" file
%  and remember to run:
% latex bibtex latex latex
% to resolve all references
%
% ACM needs 'a single self-contained file'!
%
%APPENDICES are optional
%\balancecolumns
%% \appendix
%% %Appendix A
%% \section{Headings in Appendices}
%% The rules about hierarchical headings discussed above for
%% the body of the article are different in the appendices.
%% In the \textbf{appendix} environment, the command
%% \textbf{section} is used to
%% indicate the start of each Appendix, with alphabetic order
%% designation (i.e. the first is A, the second B, etc.) and
%% a title (if you include one).  So, if you need
%% hierarchical structure
%% \textit{within} an Appendix, start with \textbf{subsection} as the
%% highest level. Here is an outline of the body of this
%% document in Appendix-appropriate form:
%% \subsection{Introduction}
%% \subsection{The Body of the Paper}
%% \subsubsection{Type Changes and  Special Characters}
%% \subsubsection{Math Equations}
%% \paragraph{Inline (In-text) Equations}
%% \paragraph{Display Equations}
%% \subsubsection{Citations}
%% \subsubsection{Tables}
%% \subsubsection{Figures}
%% \subsubsection{Theorem-like Constructs}
%% \subsubsection*{A Caveat for the \TeX\ Expert}
%% \subsection{Conclusions}
%% \subsection{Acknowledgments}
%% \subsection{Additional Authors}
%% This section is inserted by \LaTeX; you do not insert it.
%% You just add the names and information in the
%% \texttt{{\char'134}additionalauthors} command at the start
%% of the document.
%% \subsection{References}
%% Generated by bibtex from your ~.bib file.  Run latex,
%% then bibtex, then latex twice (to resolve references)
%% to create the ~.bbl file.  Insert that ~.bbl file into
%% the .tex source file and comment out
%% the command \texttt{{\char'134}thebibliography}.
%% % This next section command marks the start of
%% % Appendix B, and does not continue the present hierarchy
%% \section{More Help for the Hardy}
%% The sig-alternate.cls file itself is chock-full of succinct
%% and helpful comments.  If you consider yourself a moderately
%% experienced to expert user of \LaTeX, you may find reading
%% it useful but please remember not to change it.
%% %\balancecolumns % GM June 2007
%% % That's all folks!
\end{document}
